\begin{abstractpage}


\begin{abstract}{english}

The second-hand car market in Romania is continually growing, registering over 660.000 transactions in 2023, according to data from Autovit \cite{autovit_statistics}. However, the market expansion is shadowed by challenges, particularly the inflated prices of used cars. In response to these market dynamics, this thesis introduces a web application under the brand \textit{RoCar}, tailored for the Romanian automotive market, designed to predict vehicle prices.

\textit{RoCar} utilizes a model specialized in Romania's economy and pricing, trained on data we have independently scraped. The application differentiates itself from other similar options by using not only structured data such as the year of production, manufacturer, model, and various options but also information extracted from images and descriptions, leveraging a multimodal architecture for a more accurate prediction.

The model is accessible for inference both through a web interface and via an API, thus facilitating the usage by both individual users and businesses wishing to integrate the model into their workflows.

\end{abstract}

\begin{abstract}{romanian}

Piața autoturismelor second-hand din România este în continuă creștere, înregistrând peste 660.000 de tranzacții în anul 2023, conform datelor celor de la Autovit \cite{autovit_statistics}. Cu toate acestea, expansiunea pieței este umbrită de provocări, în special de prețurile exagerate ale mașinilor second-hand. În răspuns la aceste dinamici ale pieței, această teză introduce o aplicație web sub brand-ul \textit{RoCar}, adaptată pentru piața auto românească, proiectată pentru a prezice prețurile autovehiculelor.

\textit{RoCar} are la bază un model specializat pe economia și prețurile din România, antrenat pe date colectate independent. Aplicația se diferențiază de restul opțiunilor similare prin faptul că, pe lângă datele structurate precum anul fabricației, producătorul, modelul, și diverse opțiuni, folosește și informații extrase din imagini și descrieri, bazându-se pe o arhitectură multi-modală pentru o predicție cât mai exactă.

Modelul este accesibil pentru inferență atât prin intermediul unei interfețe web, cât și printr-un API, facilitând astfel utilizarea atât pentru utilizatori simpli, cât și pentru business-uri ce doresc integrarea modelului în flow-urile lor.

\end{abstract}

\end{abstractpage}
