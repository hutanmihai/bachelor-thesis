\chapter{Introduction}

\section{Motivation}

In today's fast-paced world, personal mobility plays a crucial role in our daily lives. Owning a car is a decision that impacts our independence, our ability to fulfill our tasks, and our overall lifestyle. One of the primary advantages of owning a car is the unparalleled convenience it offers. Unlike relying on public transportation, having your own vehicle allows you to travel on your own terms. Whether it is commuting to work, running errands, or embarking on road trips, a car provides flexibility and saves valuable time. However, there is an excellent variety of cars offered on the market, and it is essential that one makes the right choice.

With an increasing number of households owning multiple vehicles, affording new cars has become a financial burden, resulting in more people looking for previously owned cars. The inception of RoCar was driven by a personal necessity and the gaps I observed in existing tools while searching for a second-hand car. I relied on services like CarVertical \cite{carVertical} for history reports and even some price predictors to check the advertisements that caught my interest. However, I soon realized that these existing predicting tools had significant limitations. They often considered only a limited set of data points, failing to capture the comprehensive details necessary for accurate car valuation.

RoCar is rooted in the desire to provide a reliable tool that enhances the car-buying experience for individuals like myself and facilitates access to accurate car valuations for a broader audience. By addressing the shortcomings of existing instruments, RoCar seeks to empower users with the information they need to make informed decisions in the second-hand car market, thereby reducing the risk of overpaying and ensuring fair transactions.

\section{Functionality}

Our primary functionality centers around the car price prediction model, designed to emulate the human evaluation process of a second-hand car by considering all aspects, from structured details to descriptions and images.

We aim to smoothly integrate this mechanism into the typical flow of a person looking to purchase a second-hand car. The journey usually begins with browsing various forums, followed by consulting specialized websites. In Romania, online platforms like OLX \cite{OLX} and Autovit \cite{Autovit} are popular choices for prospective buyers who want to explore numerous listings. Each listing generally includes a brief overview of the car's basic specifications, a detailed description of its features, and images showcasing its condition. This initial assessment is crucial for narrowing down potential options based on specific features and overall presentation.

At this stage, users often have several cars on their watchlist but still face uncertainty regarding critical questions such as whether the car is overpriced or the right price and whether there is any room for negotiation. That would be where our application enters the flow in an effort to help improve the user experience.

Accurately determining a car's price can be challenging, even with all the necessary details and information in the advertisement. This task is incredibly daunting for everyday users who are not automotive experts or enthusiasts. Pricing complexity arises from numerous factors. For example, two identical models from the same manufacturer can differ in cost by thousands of euros due to seemingly minor features like heated seats or an advanced infotainment system. While these elements might seem relatively small, they significantly impact the vehicle's final price.

Our core product, the Car Price Predictor, directly addresses this challenge. It employs a data-driven approach to provide unbiased estimations of a car's market value. By analyzing vast amounts of data, including but not limited to model specifications, additional features, and historical pricing trends, our tool empowers users with accurate and reliable price predictions. This simplifies the decision-making process in purchasing cars, saving time and ensuring buyers can make informed, confident choices, without requiring extensive automotive knowledge. Our model effectively demystifies car pricing, making it transparent and accessible to all users.

\section{Target Audience}

The primary target audience for RoCar includes individuals who possess limited knowledge about automobiles but are looking to make informed decisions in a market that treats cars as high-value commodities. In Romania, as in many parts of the world, purchasing a car represents a significant financial commitment. However, the complexity of car valuations and the prevalence of inflated prices in the second-hand market can put inexperienced buyers at risk of overpaying or falling victim to scams.

This application is specifically designed to empower these users by providing a reliable tool for accurate car price predictions. By simplifying the car valuation process through an intuitive web interface, the app enables its users to gain critical insights into fair market prices without needing extensive automotive knowledge. This simplified process is particularly beneficial for those navigating the used car market for the first time or those who do not have the time or resources to become experts in automotive valuation.

In addition, RoCar's application is not just for individuals, as it can also be a powerful tool for car rental businesses. These companies often struggle with the evaluation in the disposal of old vehicles in the fleet. This process can often be time-consuming and inefficient, leading to vehicles behind sold well under their actual value or being left in storage for years, thus representing lost revenue. RoCar's application and API integration intend to change that. It streamlines this tedious process, enabling car rental businesses to make decisions fueled by analytical data regarding their used vehicles, thereby maximizing their revenue potential.

\section{Personal Contribution}

During the development of this thesis, we created the most extensive dataset of Romanian second-hand car listings to date. While other studies, such as the paper \textit{"Car Price Quotes Driven by Data - Comprehensive Predictions Grounded in Deep Learning Techniques} " \cite{dutulescu2023car} reference a Romanian dataset of over 15,000 car quotes, we have tripled this number in our dataset. Although our dataset, similar to others we reviewed, is not manually curated, we believe it significantly enhances the statistical insights available for future research due to its increased size.

Another essential contribution of this thesis is the multimodal approach we adopted. We hope this innovative method will pave the way for more advanced multimodal architectures in future studies on this topic.

\section{Thesis Overview}

The structure of the thesis is split into seven chapters.
\\

Chapter 1 contains the introductory part of the thesis, where we showcase our motivation for creating this application, the main functionality it provides, and present our target audience.
\\

Chapter 2 is dedicated to studies that helped us during development by providing relevant information on our progress. These studies are also divided into three different categories, where each of them is explained in more detail.
\\

Chapter 3 presents the creation of our dataset, detailing the scraping, formatting, and cleaning processes. It also provides valuable insights into our data analysis process as we try to understand the dataset in depth.
\\

Chapter 4 is the core of our thesis. It explains our incremental steps to achieving a multimodal regression model for our car price prediction application. This step and the previous chapter represent the most exciting parts of our project, where most of the work has been put in.
\\

Chapter 5 displays the application model, with both the backend and the frontend components. We provide detailed explanations of architectural decisions and how we created a framework that is easily scalable and maintainable through good coding practices and rigorous testing.
\\

Chapter 6 is the chapter that provides the conclusion of this study, where we present areas where future improvements and also underline the most significant challenges faced during the thesis development.
\\
